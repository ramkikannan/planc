\documentclass[xcolor=dvipsnames]{beamer}

\usepackage{setspace}
\usepackage{beamerthemesplit}
\usepackage{graphicx}
\usepackage{epstopdf}
\usepackage[english]{babel}
\usepackage[latin1]{inputenc}
\usepackage{times}
\usepackage[T1]{fontenc}
\usepackage{color}
\usepackage{colortbl}
%\usepackage{algorithm,algorithmic}
\usepackage{multimedia,xmpmulti}
\usepackage{multirow,multicol}
\usepackage[absolute,overlay]{textpos}
\usepackage{multimedia,xmpmulti}
\usepackage{tabularx}
\usepackage{algorithm,algpseudocode}
\usepackage{subfig}

\usepackage{pgfplots}
\usepackage{pgfplotstable}
\usepackage{etoolbox}
\usetikzlibrary{patterns}

\usepackage{ourmacros}

%% Tikz - For making pretty pictures
\usepackage{tikz}
\usetikzlibrary{3d}
\usetikzlibrary{patterns}
\usetikzlibrary{calc}
\usetikzlibrary{arrows}



\newcommand{\red}[1]{\textcolor{red}{#1}}
\newcommand{\blue}[1]{\textcolor{blue}{#1}}
\newcommand{\multiplycolor}{blue}
\newcommand{\zero}{}


\graphicspath{{fig/}}

\mode<presentation>

\definecolor{wfugold}{rgb}{0.6196078,0.494117647,0.21960784}
\usetheme{Warsaw}
\usecolortheme[named=wfugold]{structure}
%Madrid
%\usecolortheme{dolphin}
%\useinnertheme{rounded}
%\usefonttheme{serif}
\setbeamertemplate{navigation symbols}{} % gets rid of navigation bars
\setbeamertemplate{footline}
{
  \hbox{
  \begin{beamercolorbox}[wd=.33\paperwidth,ht=2.25ex,dp=1ex,left]{author in head/foot}%
    \usebeamerfont{author in head/foot}
    Ballard
  \end{beamercolorbox}%
  \begin{beamercolorbox}[wd=.34\paperwidth,ht=2.25ex,dp=1ex,center]{title in head/foot}%
    \usebeamerfont{title in head/foot}
    
  \end{beamercolorbox}%
  \begin{beamercolorbox}[wd=.33\paperwidth,ht=2.25ex,dp=1ex,right]{date in head/foot}%
    \usebeamerfont{date in head/foot}
    \insertframenumber{} \hspace*{2ex} 
  \end{beamercolorbox}}%
}

\title{Parallel Nonnegative CP Decomposition \\ of Dense Tensors}
\author[]{Grey Ballard, Koby Hayashi, Ramakrishnan Kannan}
\institute{}
\date{March 26, 2018}
%\titlegraphic{\includegraphics[scale=.15]{wfulogo}}

\begin{document}

\begin{frame}[plain]
\maketitle
\end{frame}
\addtocounter{framenumber}{-1}

\begin{frame}
\frametitle{Nonnegative CP via Block Coordinate Descent (ALS)}

\footnotesize
\begin{itemize}
	\item Basic algorithm is block coordinate descent, each block is factor matrix
	\item For each block, solve nonnegative linear least squares problem
	\begin{itemize}
		\item \scriptsize assuming least squares loss function (Gaussian distribution)
	\end{itemize}
\end{itemize}
\normalsize

\vfill

\only<1>{
%\begin{algorithm}
%\caption{$(\CPl,\epsilon) = \text{NNCP}(\TA,k)$}
%\caption{$\CP = \text{NNCP}(\TA,k)$}
%\label{alg:nncp}
\begin{algorithmic}[1]
\Require $\TA$ is $I_1\times \cdots \times I_N$ tensor, $k$ is approximation rank
%\State \Comment{Compute NNCP approximation}
\While{not converged}
	\State \Comment{Perform outer iteration of BCD}
	\For{$n=1$ to $N$}
	\State \Comment{Compute new factor matrix in $n$th mode}
	\State \red{$\Mn{M}{n} = \text{MTTKRP}(\TA,\{\Mn{H}{i}\},n)$}
	\State $\Mn{S}{n} = \Mn{G}{1} \Hada \cdots \Hada \Mn{G}{n-1} \Hada \Mn{G}{n+1} \Hada \cdots \Hada \Mn{G}{N}$
	\State \blue{$\Mn{H}{n} = \text{NLS-Update}(\Mn{S}{n},\Mn{M}{n})$}
%	\State $\Mn[\hat]{H}{n} = \text{NLS-Update}(\Mn{S}{n},\Mn{M}{n})$
%	\State \Comment{Normalize columns}
%	\State $\V{\lambda} = \text{Col-Norms}(\Mn[\hat]{H}{n})$
%	\State $\Mn{H}{n} = \text{Col-Scale}(\Mn[\hat]{H}{n},\V{\lambda})$
%	\State \Comment{Organize data for later modes}
	\State $\Mn{G}{n} = \MnTra{H}{n}\Mn{H}{n}$
	\EndFor
%	\State \Comment{Compute relative error $\epsilon$ from mode-$N$ matrices}
%	\State $\beta = \langle \Mn{M}{N},\Mn[\hat]{H}{N} \rangle$
%	\State $\gamma = \V{\lambda}^\Tra (\Mn{S}{N} \Hada \Mn{G}{N}) \V{\lambda}$
%	\State $\epsilon = \sqrt{(\alpha-2\beta+\gamma)/\alpha}$ 
\EndWhile
%\Ensure $\|\TA - \dsquare{\V{\lambda}; \Mn{H}{1},\dots,\Mn{H}{N}}\| /\|\TA\| = \epsilon$
\Ensure $\TA \approx \CP$
\end{algorithmic}
%\end{algorithm}
}
\only<2>{
\begin{itemize}
	\item Lots of ways to solve nonnegative linear least squares 
	\begin{enumerate}
		\item multiplicative updates (one element at a time, Lee-Seung)
		\item hierarchical alternating least squares (one column at a time, Cichocki et al.)
		\item block principal pivoting (active set method, Park et al.)
		\item Nesterov algorithm (used by Sidiropolous et al.)
		\item alternating directions method of multipliers
	\end{enumerate}
	\item All are bottlenecked by MTTKRP and also need Gram 
	\begin{itemize}	
		\item MTTKRP bottleneck in dense case, maybe not sparse?
	\end{itemize}
\end{itemize}
}

\end{frame}

\begin{frame}
\frametitle{Parallel Algorithm for NNCP via BCD/ALS}

\footnotesize
\textbf{Data distribution}
\begin{itemize}
	\item block distribution of tensor over $N$-D logical processor grid
	\item row-wise distribution of each factor matrix 
\end{itemize}

\vfill

\textbf{Communication steps}
\begin{itemize}
	\item All-Gather and Reduce-Scatter across processor slices for MTTKRP
	\item All-Reduce across all processors for Gram
	\item All-Reduce across all processors for normalization and error comp.
\end{itemize}

\vfill

\textbf{Local Computation}
\begin{itemize}
	\item local MTTKRP (which we do using dimension trees)
	\item local NLS solves/updates (which we do using block principal pivoting)
	\item normalization and error computation (very cheap)
\end{itemize}


\end{frame}

% !TEX root = ../paper.tex

\newcommand{\procdim}{3}
\newcommand{\proc}{\draw[black,shift={(-.5,-.5)}] (0,0) grid (\procdim,\procdim);}
\newcommand{\highlight}{gray!75}
\newcommand{\commhighlight}{gray!25}
\newcommand{\parscale}{.46}
\newcommand{\secfaclabel}{$\Mn{M}{2}$}

\newcommand{\parbasepic}{
% highlight one processor's data
% front face
\begin{scope}[canvas is yz plane at x=.5,shift={(1.5,-1.5)}]
	% highlight front face of tensor block
	\draw[fill=\highlight,shift={(0,1)}] (0,0) rectangle (1,1);
	% highlight block of 1st factor matrix
	\draw[fill=\highlight,shift={(-2.5,2-1/3)},xscale=.5] (0,0) rectangle (-1,-1/9);
	% highlight block of 2nd factor matrix
	\draw[shift={(0,-1.5)},yscale=.5] (0,0) rectangle (1/9,-1);
\end{scope}
% right face
\begin{scope}[canvas is zx plane at y=(\procdim-.5),rotate=-90,shift={(-.5,-3)}]
	% highlight front face of tensor block
	\draw[fill=\highlight,shift={(0,2.5)}] (0,0) rectangle (1,1);
	% highlight block of 3rd factor matrix
	\draw[fill=\highlight,yscale=.5] (0,0) rectangle (1/9,-1);
\end{scope}
% top face
\begin{scope}[canvas is yx plane at z=.5,yscale=-1,rotate=0]
	% highlight front face of tensor block
	\draw[fill=\highlight,shift={(1.5,-.5)}] (0,0) rectangle (1,1);
\end{scope}

% draw tensor
% front face
\begin{scope}[canvas is yz plane at x=.5,rotate=-90]
	\proc
\end{scope}
% top face
\begin{scope}[canvas is yx plane at z=.5,yscale=-1,rotate=0]
	\proc
\end{scope}
% right face
\begin{scope}[canvas is zx plane at y=(\procdim-.5),rotate=180]
	\proc
\end{scope}

% draw factor matrices
% front face
\begin{scope}[canvas is yz plane at x=.5,shift={(1.5,-.5)}]
	% draw 1st factor matrix
	\draw[shift={(-2.5,0)},xscale=.5] (0,-2) grid (-1,1);
	\node[draw=none] at (-3.6,-.5) {\Large $\Mn{H}{1}$};
	% draw 2nd factor matrix
	\draw[shift={(0,-2.5)},yscale=.5] (-2,0) grid (1,-1);
	\node[draw=none] at (-.5,-3.5) {\Large \secfaclabel};
\end{scope}
% right face
\begin{scope}[canvas is zx plane at y=(\procdim-.5),rotate=-90,shift={(1.5,-.5)}]
	% draw 2nd factor matrix
	\draw[shift={(0,-2.5)},yscale=.5] (-2,0) grid (1,-1);
	\node[draw=none] at (-.5,-3.5) {\Large $\Mn{H}{3}$};
\end{scope}
}

\begin{frame}
\frametitle{Inner Iteration of Parallel Algorithm}

\begin{center}
\scalebox{1.5}{
\only<1>{\input{../sc18/fig/AlgStart.tex}}
\only<2>{% !TEX root = Alg_header.tex

\begin{tikzpicture}[x={(-0.5cm,-0.4cm)}, y={(1cm,0cm)}, z={(0cm,1cm)},every node/.append style={transform shape},scale=\parscale]

% highlight all-gather/reduce-scatter patterns

% local computation
% front face
\begin{scope}[canvas is yz plane at x=.5]
	% highlight block of 1st factor matrix now owned by processor
	\draw[fill=\highlight,shift={(-1,.5)},xscale=.5] (0,0) rectangle (-1,-1);
\end{scope}
% right face
\begin{scope}[canvas is zx plane at y=(\procdim-.5),rotate=-90]
	% highlight block of 3rd factor matrix now owned by processor
	\draw[fill=\highlight,yscale=.5,shift={(-.5,-6)}] (0,0) rectangle (1,-1);
\end{scope}

% front face
\begin{scope}[canvas is yz plane at x=.5]
	% highlight block of 2nd factor matrix contribution computed by processor
	\draw[fill=\highlight,shift={(1.5,-3)},yscale=.5] (0,0) rectangle (1,-1);
\end{scope}

\parbasepic

\end{tikzpicture}

}
\only<3>{% !TEX root = Alg_header.tex

\begin{tikzpicture}[x={(-0.5cm,-0.4cm)}, y={(1cm,0cm)}, z={(0cm,1cm)},every node/.append style={transform shape},scale=\parscale]

% highlight all-gather/reduce-scatter patterns
% reduce-scatter in 2nd mode

% front face
\begin{scope}[canvas is yz plane at x=.5]
	% highlight front face of proc comm
	\draw[fill=\commhighlight,shift={(1.5,.5)}] (0,0) rectangle (1,-3);
	% highlight block of 2nd factor matrix involved in reduce scatter
	\draw[fill=\commhighlight,shift={(1.5,-3)},yscale=.5] (0,0) rectangle (1,-1);
	% highlight block of 2nd factor matrix of computed output
	\draw[fill=\highlight,shift={(1.5,-3)},yscale=.5] (0,0) rectangle (1/9,-1);
	% highlight block of 1st factor matrix now owned by processor
	\draw[fill=\highlight,shift={(-1,.5)},xscale=.5] (0,0) rectangle (-1,-1);
\end{scope}
% right face
\begin{scope}[canvas is zx plane at y=(\procdim-.5),rotate=-90]
	% highlight right face of proc comm
	\draw[fill=\commhighlight,shift={(-.5,.5)}] (0,0) rectangle (3,-3);
	% highlight block of 3rd factor matrix now owned by processor
	\draw[fill=\highlight,yscale=.5,shift={(-.5,-6)}] (0,0) rectangle (1,-1);
\end{scope}
% top face
\begin{scope}[canvas is yx plane at z=.5,yscale=-1,rotate=0]
	% highlight top face of proc comm
	\draw[fill=\commhighlight,shift={(1.5,-.5)}] (0,0) rectangle (1,3);
\end{scope}

\parbasepic

\end{tikzpicture}

}
\only<4>{\input{../sc18/fig/AlgNLS.tex}}
\only<5>{\input{../sc18/fig/AlgAG.tex}}
}

\vfill

\only<1>{Start $n$th iteration with redundant subset of rows \\ of each input matrix.}
\only<2>{Compute local MTTKRP for contribution to output matrix $\Mn{M}{2}$.}
\only<3>{Reduce-Scatter to compute and distribute rows of $\Mn{M}{2}$.}
\only<4>{Compute local NLS update to obtain $\Mn{H}{2}_{\V{p}}$ from $\Mn{M}{2}_{\V{p}}$ (and $\Mn{S}{2}$).}
\only<5>{All-Gather to collect rows of $\Mn{H}{2}$ needed \\ for later inner iterations.}

\end{center}


\end{frame}

\begin{frame}
\frametitle{Inner Iteration of Parallel Algorithm}

\begin{figure}
\centering
  \subfloat[Start]{\input{../sc18/fig/AlgStart.tex}} \quad
  \subfloat[Local MTTKRP]{% !TEX root = Alg_header.tex

\begin{tikzpicture}[x={(-0.5cm,-0.4cm)}, y={(1cm,0cm)}, z={(0cm,1cm)},every node/.append style={transform shape},scale=\parscale]

% highlight all-gather/reduce-scatter patterns

% local computation
% front face
\begin{scope}[canvas is yz plane at x=.5]
	% highlight block of 1st factor matrix now owned by processor
	\draw[fill=\highlight,shift={(-1,.5)},xscale=.5] (0,0) rectangle (-1,-1);
\end{scope}
% right face
\begin{scope}[canvas is zx plane at y=(\procdim-.5),rotate=-90]
	% highlight block of 3rd factor matrix now owned by processor
	\draw[fill=\highlight,yscale=.5,shift={(-.5,-6)}] (0,0) rectangle (1,-1);
\end{scope}

% front face
\begin{scope}[canvas is yz plane at x=.5]
	% highlight block of 2nd factor matrix contribution computed by processor
	\draw[fill=\highlight,shift={(1.5,-3)},yscale=.5] (0,0) rectangle (1,-1);
\end{scope}

\parbasepic

\end{tikzpicture}

} \quad
  \subfloat[Reduce-Scatter]{% !TEX root = Alg_header.tex

\begin{tikzpicture}[x={(-0.5cm,-0.4cm)}, y={(1cm,0cm)}, z={(0cm,1cm)},every node/.append style={transform shape},scale=\parscale]

% highlight all-gather/reduce-scatter patterns
% reduce-scatter in 2nd mode

% front face
\begin{scope}[canvas is yz plane at x=.5]
	% highlight front face of proc comm
	\draw[fill=\commhighlight,shift={(1.5,.5)}] (0,0) rectangle (1,-3);
	% highlight block of 2nd factor matrix involved in reduce scatter
	\draw[fill=\commhighlight,shift={(1.5,-3)},yscale=.5] (0,0) rectangle (1,-1);
	% highlight block of 2nd factor matrix of computed output
	\draw[fill=\highlight,shift={(1.5,-3)},yscale=.5] (0,0) rectangle (1/9,-1);
	% highlight block of 1st factor matrix now owned by processor
	\draw[fill=\highlight,shift={(-1,.5)},xscale=.5] (0,0) rectangle (-1,-1);
\end{scope}
% right face
\begin{scope}[canvas is zx plane at y=(\procdim-.5),rotate=-90]
	% highlight right face of proc comm
	\draw[fill=\commhighlight,shift={(-.5,.5)}] (0,0) rectangle (3,-3);
	% highlight block of 3rd factor matrix now owned by processor
	\draw[fill=\highlight,yscale=.5,shift={(-.5,-6)}] (0,0) rectangle (1,-1);
\end{scope}
% top face
\begin{scope}[canvas is yx plane at z=.5,yscale=-1,rotate=0]
	% highlight top face of proc comm
	\draw[fill=\commhighlight,shift={(1.5,-.5)}] (0,0) rectangle (1,3);
\end{scope}

\parbasepic

\end{tikzpicture}

} \quad
  \subfloat[Local NLS]{\input{../sc18/fig/AlgNLS.tex}} \quad
  \subfloat[All-Gather]{\input{../sc18/fig/AlgAG.tex}}
\end{figure}
\end{frame}

\begin{frame}
\frametitle{Dimension Trees}

\begin{center}
\footnotesize
Dimension trees systematically avoid recomputation across MTTKRPs 
\normalsize

\vfill

% !TEX root = ../paper.tex

\begin{center}
\begin{tikzpicture}

\node (12345) at (4,2) {$\{1,2,3,4,5\}$};
\node (12) at (2,1) {$\{1,2\}$};
\node (345) at (6,1) {$\{3,4,5\}$};
\node (1) at (1,0) {$\{1\}$};
\node (2) at (3,0) {$\{2\}$};
\node (3) at (5,0) {$\{3\}$};
\node (45) at (7,0) {$\{4,5\}$};
\node (4) at (6,-1) {$\{4\}$};
\node (5) at (8,-1) {$\{5\}$};

\scriptsize
\path[draw] (12345) edge [left,align=right] node {PM} (12);
\path[draw] (12345) edge [right,align=left] node {PM} (345);
\path[draw] (12) edge [left,align=right] node {mTTV} (1);
\path[draw] (12) edge [right,align=left] node {mTTV} (2);
\path[draw] (345) edge [left,align=right] node {mTTV} (3);
\path[draw] (345) edge [right,align=left] node {mTTV} (45);
\path[draw] (45) edge [left,align=right] node {mTTV} (4);
\path[draw] (45) edge [right,align=left] node {mTTV} (5);
\normalsize

\end{tikzpicture}
\end{center}
\end{center}

\vfill

\footnotesize
Nearly all the time spent in partial MTTKRP (PM), same number of flops as MTTKRP but easier to implement/parallelize (one DGEMM call)
\normalsize

\end{frame}

\begin{frame}
\frametitle{Error Computation}

Given data tensor $\TA$ and CP model $\T{M} = \CP$, we want to compute the squared error
$$\| \TA - \T{M}\|^2 = \|\TA\|^2 - 2\langle \TA, \T{M} \rangle + \|\T{M}\|^2$$
which we do efficiently using the identity
$$\langle \TA, \T{M} \rangle = \langle \Mn{M}{N}, \Mn{H}{N} \rangle$$ 
where 
$$\Mn{M}{N} = \Mz{A}{N} (\Mn{H}{N-1} \Khat \cdots \Khat \Mn{H}{1})$$

\vfill

*This technique is noted in a paper by Smith and Karypis and used in code by Phan, Tichavsky, and Cichocki

\end{frame}

\begin{frame}
\frametitle{Comparison against existing code}

From paper: \\
\footnotesize
A. P. Liavas, G. Kostoulas, G. Lourakis, K. Huang, and N. D. Sidiropoulos, "Nesterov-based Alternating Optimization for Nonnegative Tensor Factorization: Algorithm and Parallel Implementations", \emph{IEEE Transactions on Signal Processing}, vol. 66, no. 4, Feb. 2018.

\vfill

Code available: \url{http://www.telecom.tuc.gr/Greek/Liavas/publications/MPI_EIGEN_Nesterov_NTF.zip}

\vfill

Not a whole lot of detail in paper about implementation, but data distribution and communication pattern the same as ours
\begin{itemize}
	\item limited to 3D tensors
	\item uses Eigen for local dense linear algebra
	\item no use of dimension trees
	\item inefficient error computation
\end{itemize}

\end{frame}

% !TEX root = experiments.tex

% macros for plotting
\newcommand{\datafile}{}
\newcommand{\alg}{}
\newcommand{\numiterations}{30}
\newcommand{\minvalue}{1}

% toggles whether or not to plot naive results (if not, the rows of the data file also need to be commented out)
\newif\ifnaive
% toggles ksweep vs scaling plot
\newif\ifksweep
% toggles legend
\newif\iflegend
% toggles y axis label
\newif\ifylabel
% toggles wider plot area
\newif\ifwider


\newcommand{\setcolors}{
\pgfplotsset{cycle list={
	red, fill=red \\ 
	blue, fill=blue \\ 
	green, fill=green \\ 
	red, pattern=crosshatch, pattern color=red \\
	blue, pattern=crosshatch, pattern color=blue \\
	green, pattern=crosshatch, pattern color=green \\
}};
}

% set options for grouped bar plot
\newcommand{\breakdownplotoptions}{
	ybar stacked,
	reverse legend,
	bar width=6pt,
	width=9cm, height=3.85cm,
	ylabel={Time (seconds)}, 
	y label style={yshift=-.5cm},
	ymin=0,
	symbolic x coords={1,16,81,256},
	xtick=data,
	legend={DimTree}
}

% stacked bar plot command
\newcommand{\breakdownplot}{
\begin{axis}[\breakdownplotoptions]
	\setcolors
	\addplot table[x=p, y expr=(\thisrow{mttkrp}/(\minvalue*\numiterations))] {\datafile};
	\addplot table[x=p, y expr=(\thisrow{nnls}/(\minvalue*\numiterations))] {\datafile};
	\addplot table[x=p, y expr=(\thisrow{gram}/(\minvalue*\numiterations))] {\datafile};
	\addplot table[x=p, y expr=(\thisrow{allgather}/(\minvalue*\numiterations))] {\datafile};
	\addplot table[x=p, y expr=(\thisrow{reducescatter}/(\minvalue*\numiterations))] {\datafile};
	\addplot table[x=p, y expr=(\thisrow{allreduce}/(\minvalue*\numiterations))] {\datafile};
\end{axis}
}

% labels for bar groups (manually positioned)
\newcommand{\labels}{
%\node [align=center,text width=3cm] at (1.25cm, -.95cm)   {\ifksweep 10 \else 16 \fi};
%\node [align=center,text width=3cm] at (3cm, -0.95cm)   {\ifksweep 20 \else 96 \fi};
%\node [align=center,text width=3cm] at (4.6cm, -0.95cm)   {\ifksweep 30 \else 384 \fi};
%\node [align=center,text width=3cm] at (6.3cm, -0.95cm) {\ifksweep 40 \else 864 \fi};
%\node [align=center,text width=3cm] at (8.15cm, -0.95cm) {\ifksweep 50 \else 1536 \fi};
\node [align=center,text width=3cm] at (1cm, -1.15cm)   {\ifksweep 10 \else 16 \fi};
\node [align=center,text width=3cm] at (2.4cm, -1.15cm)   {\ifksweep 20 \else 96 \fi};
\node [align=center,text width=3cm] at (3.7cm, -1.15cm)   {\ifksweep 30 \else 384 \fi};
\node [align=center,text width=3cm] at (5.1cm, -1.15cm) {\ifksweep 40 \else 864 \fi};
\node [align=center,text width=3cm] at (6.4cm, -1.15cm) {\ifksweep 50 \else 1536 \fi};
}

% allows for filtering rows from dat file
\pgfplotsset{
    discard if not/.style 2 args={
        x filter/.append code={
            \edef\tempa{\thisrow{#1}}
            \edef\tempb{#2}
            \ifx\tempa\tempb
            \else
                \def\pgfmathresult{inf}
            \fi
        }
    }
}

% set options for strongscaling plot
\newcommand{\strongscalingplotoptions}{
	log basis y={2},
	log basis x={2},
	xlabel=Nodes,
	ylabel=Time (s),
	y tick label style={
        /pgf/number format/.cd,
            precision=4,
	},
	legend style={draw=none, cells={align=left}, nodes={scale=0.7}}
}

% plot time over p, filtering rows out appropriately
\newcommand{\strongscalingplot}{
\begin{loglogaxis}[\strongscalingplotoptions]
	%\addplot+ [discard if not={algo}{liavas},thick,mark options={solid}] table [x={nodes}, y={totaltime}] {\datafile};
	%\addplot+ [discard if not={algo}{nodimtree},thick,mark options={solid},mark=square*] table [x={nodes}, y={totaltime}] {\datafile};
	\addplot+ [discard if not={alg}{D},thick,mark options={solid},mark=triangle*] table [x={p}, y={total}] {\datafile};
	\legend{DimTree}
\end{loglogaxis}
}



\begin{frame}
\frametitle{Strong Scaling on 3D Tensor}

\begin{figure}
\begin{tikzpicture}[scale=.9]
\renewcommand{\datafile}{../sc18/data/str_3D_syn.dat}
\renewcommand{\numiterations}{42}
\liavastrue
\strongscalingplot
\end{tikzpicture}
\caption{Strong scaling of 3D synthetic tensor with dimension $1024\times 1024\times 1024$ on processor grids $2^k\times 2^k\times 2^k$ for $k\in\{0,1,2,3\}$.  The rank is fixed at 32.}
\end{figure}

\end{frame}

\begin{frame}
\frametitle{Strong Scaling on 5D Tensor}

\begin{figure}
\begin{tikzpicture}[scale=.9]
\renewcommand{\datafile}{../sc18/data/str_5D_syn.dat}
\renewcommand{\numiterations}{10}
\liavasfalse
\strongscalingplot
\end{tikzpicture}
\caption{Strong scaling of 5D synthetic tensor with dimension $64\times 64\times 64\times 64\times 64$ on processor grids $1\times1\times1\times1\times1$, $2\times1\times1\times1\times1$, $\dots$, $2\times2\times2\times2\times2$.  The rank is fixed at 32.}
\end{figure}

\end{frame}

\begin{frame}
\frametitle{Weak Scaling Time Breakdown}

\begin{figure}
\begin{tikzpicture}
\renewcommand{\datafile}{../sc18/data/wk_4D_syn.dat}
\renewcommand{\numiterations}{10}
\breakdownplot
%\labels
\end{tikzpicture}
\caption{Weak scaling of 4D synthetic tensors with (D) and without (N) the use of dimension trees.  The tensor and processor grid dimensions are $128k\times 128k\times 128k\times 128k$ and $k\times k\times k\times k$ for $k\in\{1,2,3,4\}$, and the rank is fixed at 32.}
\end{figure}

\end{frame}

\begin{frame}
\frametitle{Varying Rank Time Breakdown}

\begin{figure}
\renewcommand{\datafile}{../sc18/data/ksw_4D_HSI.dat}
\renewcommand{\numiterations}{10}
\begin{tikzpicture}[scale=.8]
\begin{axis}[	
	ybar stacked,
	bar width=12pt,
	width=\columnwidth,
	height =.5\columnwidth,
	%width=9cm, height=3.85cm,
	ylabel={Time (s)}, 
	xlabel={Rank $k$},
	y label style={yshift=-.5cm},
	ymin=0,
	%symbolic x coords={D10,N10,,D20,N20,,D30,N30,,D40,N40,,D50,N50},
	symbolic x coords={D10,D20,D30,D40,D50},
	%xtick=data,
	xticklabels={,10,,20,,30,,40,,50},
	legend style={at={(0.5,1.3)},anchor=north},
	legend columns=-1,
]
	\setcolors
	\addplot table[x=alg-K, y expr=(\thisrow{mttkrp}/(\minvalue*\numiterations))] {\datafile};
	\addplot table[x=alg-K, y expr=(\thisrow{mttv}/(\minvalue*\numiterations))] {\datafile};
	\addplot table[x=alg-K, y expr=((\thisrow{nnls}+\thisrow{gram})/(\minvalue*\numiterations))] {\datafile};
	\addplot table[x=alg-K, y expr=((\thisrow{allgather}+\thisrow{reducescatter})/(\minvalue*\numiterations))] {\datafile};
	\addplot table[x=alg-K, y expr=(\thisrow{allreduce}/(\minvalue*\numiterations))] {\datafile};
	\legend{PM,mTTV,NLS,Factor Comm,Gram Comm};
\end{axis}
%\labels
\end{tikzpicture}
\caption{Per-iteration time breakdown of our implementation (using dimension trees) over various ranks for a time-lapse HSI dataset with dimensions $1344\times 1024\times 33 \times 9$ on 64 processors arranged in a $8\times8\times1\times1$ grid.}
\end{figure}

\end{frame}

\begin{frame}
\frametitle{Conclusions (so far)}

\footnotesize
\begin{enumerate}
	\item Hard not to be compute bound in dense case
	\begin{itemize}
	\scriptsize
		\item computation is dominated by $O(R \prod I_k / P_k)$
		\item communication is dominated by $O(R \sum I_k / P_k)$
		\item computation scales with $P$, communication scales with $P^{1/N}$, but it doesn't matter until $P$ is huge
	\end{itemize}
	\vfill
	\item Dimension trees make huge difference for higher-order tensors
	\begin{itemize}
	\scriptsize
		\item dominant computation term predicts $N/2$ speedup
		\item avoiding large Khatri-Rao products also makes a big difference
		\item DGEMM calls have more favorable shapes
	\end{itemize}
	\vfill
	\item Liavas et al.'s code has some shortcomings
	\begin{itemize}
	\scriptsize
		\item limited to 3D tensors
		\item stores each matricization separately (each in col-major)
		\item does not use dimension trees or efficient error computation
		\item this doesn't explain everything, but we don't have a time breakdown for further detail
	\end{itemize}
\end{enumerate}

\end{frame}

\end{document}

